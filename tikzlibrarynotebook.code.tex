\typeout{TikzNotebook}
\RequirePackage{pgfkeys}
\usepackage[strict]{changepage}
\usepackage{etoolbox}
\usepackage{lastpage}
\usetikzlibrary{calc,patterns,patterns.meta,fit,calendar,shapes.misc}
\usepackage{tikzpagenodes}

\newsavebox{\notebookelembox}

\newdimen\elemwidth
\newif\ifelem@top

\def\nbkeys#1{\pgfkeys{/notebook/.cd,#1}}

\makeatletter

\def\pgfkeysdefargopt#1#2#3{%
  \long\def\pgfkeys@temp{%
    \@ifnextchar[%
    {\pgfkeysvalueof{#1/.@body}}%
    {\pgfkeysvalueof{#1/.@body}[#2]}%
  }%
  \pgfkeyslet{#1/.@cmd}{\pgfkeys@temp}%
  \def\pgfkeys@temp[##1]##2\pgfeov{#3}%
  \pgfkeyslet{#1/.@body}{\pgfkeys@temp}%
}


\pgfkeysdefnargs{/handlers/.code 2 opt}{2}{
  \edef\pgfkeys@temp{\pgfkeyscurrentpath}%
  \expandafter\pgfkeysdefargopt\expandafter{\pgfkeys@temp}{#1}{#2}}

\pgfkeys{/handlers/.style 2 opt/.code 2 args=\pgfkeys{\pgfkeyscurrentpath/.code 2 opt={#1}{\pgfkeysalso{#2}}}}


\nbkeys{
  canvas start/.code={%
    \node[inner sep=0,outer sep=0,fit=(#1), name=canvas]{};
    \pgfpointanchor{canvas}{east}%
    \elemwidth=\pgf@x%
    \pgfpointanchor{canvas}{west}%
    \addtolength{\elemwidth}{-\pgf@x}
    \node[inner sep=0,outer sep=0,minimum width=\elemwidth,minimum height=0, anchor=north east] at (canvas.north east) (top-elem) {};
    \node[inner sep=0,outer sep=0,minimum width=\elemwidth,minimum height=0, anchor=south east] at (canvas.south east) (bottom-elem) {};
    \nbkeys{every canvas/.try}
  },
   % Page Elements: Attached to current page or canvas.
   paginate format/.style={font=\scriptsize\itshape\color{gray}},
   paginate/.code={
     \node[anchor=north,/notebook/paginate format] at (canvas.south){#1};
   },
   paginate/.default={\thepage},
   foldmark/.code={
     \ifoddpage
       \draw[] (current page.west) -- ++(east:#1);
     \else
       \draw[] (current page.east) -- ++(west:#1);
     \fi
   },
   foldmark/.default=8mm,
   %
   punchmark/.style={
     ultra thin,circle,draw=gray!50,inner sep=0,anchor=center,minimum size=#1
   },
   punchmarks/.code 2 opt={6mm}{
     \foreach \offset in {#2} {
       \ifoddpage
         \path (current page.west)++(east:9mm)++(up:\offset) node[/notebook/punchmark=#1]{};
         \path (current page.west)++(east:9mm)++(down:\offset) node[/notebook/punchmark=#1]{};
       \else
         \path (current page.east)++(west:9mm)++(up:\offset) node[/notebook/punchmark=#1]{};
         \path (current page.east)++(west:9mm)++(down:\offset) node[/notebook/punchmark=#1]{};
       \fi
     }
   },
   % Space Fill Elements
   every pattern/.style={
     pattern color=gray
   },
  fill elem/.code 2 opt={}{
    \path [#2] (top-elem.south west) rectangle (bottom-elem.north east);
    \node[fit=(top-elem.south west) (bottom-elem.north east),outer sep=0pt,inner sep=0pt,#1] (top-elem){};
    \node[fit=(top-elem),inner sep=0pt] (elem){};
  },
  fill dotted/.style 2 opt={}{
    fill elem={[#1]
      /notebook/every pattern,pattern={Dots[distance=0.25in,radius=.8pt,#2]}
    }
  },
  fill squares/.style 2 opt={}{
    fill elem={[#1]
      /notebook/every pattern,pattern={Hatch[distance=5mm,#2]}
    }
  },
  fill lines/.style 2 opt={}{
    fill elem={[#1]
      /notebook/every pattern,pattern={Lines[distance=9mm,#2]}
    }
  },
  % Filled Elements
  elem/.code 2 opt={}{
    \begin{lrbox}{\notebookelembox}
      \begin{pgfinterruptpicture}
        \begin{tikzpicture}[remember picture,/notebook/every elem content]
          \path (0,0);
          #2
        \end{tikzpicture}
      \end{pgfinterruptpicture}
    \end{lrbox}
    \node[/notebook/elem style/.cd,outer sep=0,inner sep=0pt,top,
          append after command={
            \pgfextra{\node[fit=(\tikzlastnode),name=elem,inner sep=0pt,outer sep=0pt] {};}
          },/notebook/every border,#1] {\usebox{\notebookelembox}};
  },
  /notebook/elem style/.search also={/tikz},
  /notebook/elem style/.cd,
  top/.style={
    anchor=north west, at=(top-elem.south west),name=top-elem,@top=true,
  },
  bottom/.style={
    anchor=south west, at=(bottom-elem.north west),name=bottom-elem,@top=false,
  },
  @top/.is if=elem@top,
  inner sep/.style={
    /tikz/text width=\elemwidth-2*#1,
    /tikz/inner sep=#1,
  },
  top rule/.style={
    append after command={
      \pgfextra{\draw[/notebook/every border,#1] (\tikzlastnode.north west) -- (\tikzlastnode.north east);}
    },
  },
  bottom rule/.style={
    append after command={
      \pgfextra{\draw[/notebook/every border,#1] (\tikzlastnode.south west) -- (\tikzlastnode.south east);}
    },
  },
  draw/.style={
    yshift=\ifelem@top\pgflinewidth\else-\pgflinewidth\fi,
    /tikz/draw=#1,
    append after command={
      \pgfextra{
        \node[fit=(\tikzlastnode),name=\tikzlastnode,outer sep=0,inner ysep=\pgflinewidth,inner xsep=0] {};
      }
    }
  },
  /notebook/.cd,
  % Decoration elements
  % FIXME: Make these elem style
  every border/.style={
    color=gray,
  },
  every elem content/.style={
    color=black,
  },
  every label/.style={
    fill=red,
    align=center,
    font=\itshape\color{gray},
  },
  label/.code 2 args={
    \node[anchor=#1,outer ysep=2\pgflinewidth,/notebook/every label] at (elem.#1) {#2};
  },
  time codes/.code={
    \fill[gray!30](elem.north west) rectangle ($(elem.south west)+(east:1.2em)$);
    \foreach \time/\pos in {08/0.0,09/0.1,10/0.2,11/0.3,12/0.4,13/0.5,14/0.6,15/0.7,16/.8,17/0.9} {
      \path ($(elem.north west)+(east:0.6em)$) coordinate(@)
          -- node[pos=\pos,anchor=north, inner sep=2pt] {\color{gray}\scriptsize \time}
       (@|-elem.south);
    }
  },
  days codes/.code={
    \fill[gray!30](elem.north west) rectangle ($(elem.south west)+(east:1.2em)$);
    \foreach \time/\pos in {Monday/0.0,Tuesday/0.25,Wednesday/0.5,Thuusda/0.75} {
      \path ($(elem.north west)+(east:0.6em)$) coordinate(@)
          -- node[pos=\pos,anchor=north, inner sep=2pt] {\color{gray}\scriptsize \time}
       (@|-elem.south);
    }
  },
  % Special Elements
  box/.style 2 opt={draw}{
    elem={[#1]{
      \node[text width=\elemwidth-2ex,inner sep=1ex] {#2};
    }}
  },
  todo/.style 2 opt={draw}{%
    elem={[inner sep=3pt,#1]{
      \foreach \y in {1,...,#2} {
        \foreach \x in {0,1} {
          \node[draw, minimum size=4mm,anchor=north west] at (\elemwidth/2*\x,7.5mm-7.5mm*\y) (dot){};
          \draw[draw opacity=0.5] (dot.0) -- (dot.180);
          \draw[dashed] (dot.south east)++(east:1mm) -- ++(east:\elemwidth/2-6mm-6pt);
        }
      }}
    },
  },
  column divider/.store in=\columnDivider,
  column divider=0pt,
  columns/.style 2 opt={draw}{
    elem={[#1]{
      \coordinate (@@) at (0,0);
      \def\notfirst##1{}
      \foreach \field/\cols in {#2} {
        \node[outer sep=0, anchor=north west,align=left,minimum width=(\elemwidth)/6*\cols,at=(@@.north east)]  (@@) {\strut};
        \node[anchor=base west] at (@@.base west) {\strut \field};
        \notfirst{
          \draw[/notebook/every border] ($(@@.north west)+(west:.5\pgflinewidth)+(down:\columnDivider)$)
          -- ($(@@.south west)+(west:.5\pgflinewidth)+(up:\columnDivider)$);
        }
        \global\let\notfirst\relax
      }
    }},
  },
}

% Crosses
\tikzdeclarepattern{
  name=Crosses,
  type=uncolored,
  bounding box={(-5pt,-5pt) and (5pt,5pt)},
  tile size={(\tikztilesize,\tikztilesize)},
  parameters={\tikzstarradius,\tikztilesize,\tikzlw},
  tile transformation={rotate=\tikzstarrotate},
  defaults={
    radius/.store in=\tikzstarradius,radius=1pt,
    rotate/.store in=\tikzstarrotate,rotate=0,
    distance/.store in=\tikztilesize,distance=5mm,
    line width/.store in=\tikzlw,line width=0.4pt,
  },
  code={
    \draw[line width=\tikzlw] (90:\tikzstarradius) -- (-90:\tikzstarradius);
    \draw[line width=\tikzlw] (180:\tikzstarradius) -- (0:\tikzstarradius);
  }
}
\pgfkeys{
  /notebook/fill crosses/.style={
    fill elem={
      /notebook/every pattern,pattern={Crosses[distance=5mm]}
    }
  }
}

%%%%%%%%%%%%%%%%%%%%%%%%%%%%%%%%%%%%%%%%%%%%%%%%%%%%%%%%%%%%%%%%
% Calendar
\newcount\@tempcntc

\def\@rawjuliantocalendarweek#1{%
  \@tempcnta=#1\relax%
  \pgfcalendarjuliantodate{#1}{\@tempa}{\@tempb}{\@tempc}%
  \pgfcalendardatetojulian{\@tempa-01-01+-1}{\@tempcntb}%
  \advance\@tempcnta by-\@tempcntb\relax% @tempcnta = ordinal date of start date
  \pgfcalendarjuliantoweekday{#1}{\@tempcntb}%
  \advance\@tempcntb by1\relax% @tempcntb = week day number of start date
  \pgfmathparse{int((\@tempcnta - \@tempcntb + 10) / 7)}%
}

\def\pgfcalendarjuliantocalendarweek#1#2{%
  \begingroup%
  \@rawjuliantocalendarweek{#1}%
  \@tempcnta=\pgfmathresult\relax%
  \ifnum\@tempcnta=0\relax%
    \pgfcalendarjuliantodate{#1}{\@tempa}{\@tempb}{\@tempc}%
    \pgfcalendardatetojulian{\@tempa-01-01+-1}{\@tempcntc}%
    \@rawjuliantocalendarweek{\@tempcntc}%
  \else\ifnum\@tempcnta=53\relax%
    \pgfcalendarjuliantodate{#1}{\@tempa}{\@tempb}{\@tempc}%
    \pgfcalendarifdate{\@tempa-12-31}{Thursday,Friday,Saturday,Sunday}{%
      \relax%
    }{%
      \def\pgfmathresult{1}%
    }%
  \fi\fi%
  \global#2=\pgfmathresult\relax%
  \endgroup%
}


\newcount\notebook@date
\newcount\notebook@wdcount
\newcommand{\pgfcalendarcalc}[3][]{%
  \pgfcalendardatetojulian{#2}{\notebook@date}%
  \pgfcalendarjuliantodate{\notebook@date}{\@@Y}{\@@M}{\@@D}%
  \pgfcalendarjuliantoweekday{\notebook@date}{\notebook@wdcount}%
  \xdef#3{\@@Y-\@@M-\@@D}%
  \csxdef{#1Y}{\@@Y}%
  \csxdef{#1M}{\@@M}%
  \csxdef{#1D}{\@@D}%
  \csxdef{#1WD}{\the\notebook@wdcount}
}

\newbox\calendarbox
\def\nb@calendar@kwcode{
  \pgfcalendarjuliantocalendarweek{\pgfcalendarcurrentjulian}{\notebook@wdcount}
  \node[anchor=base west,xshift=-5em-\pgfcalendarcurrentweekday*3.5ex]
     (notebook@calendar@kw)
     {\nbkeys{calendar/calendar week=\the\notebook@wdcount}};
   }

\tikzoption{day headings}{\tikzstyle{day heading}=[#1]}
\tikzstyle{day heading}=[]
\tikzstyle{day letter headings}=[
    execute before day scope={ \ifdate{day of month=1}{%
      \pgfmathsetlength{\pgf@ya}{\tikz@lib@cal@yshift}%
      \pgfmathsetlength\pgf@xa{\tikz@lib@cal@xshift}%
      \pgftransformyshift{-\pgf@ya}
      \foreach \d/\l in {0/M,1/T,2/W,3/T,4/F,5/S,6/S} {
        \pgf@xa=\d\pgf@xa%
        \pgftransformxshift{\pgf@xa}%
        \pgftransformyshift{\pgf@ya}%
        \node[every day,day heading]{\l};%
      } 
    }{}%
  }%
]
   
\pgfkeys{
  /notebook/calendar/next week/.style={fill=gray!20},
  /notebook/calendar/past/.style={strike out,draw},
  /notebook/calendar/weekend/.style={gray},
  /notebook/calendar/every calendar/.style={
    day heading/.style={font=\bfseries},
    week list,
    month label above centered,
    month text=\mbox{\textit{\%mt} \%y-},
    every day/.append style={text width=2ex,align=center},
    day letter headings,
  },
  /notebook/calendar/calendar week/.code={
    \itshape\color{gray}W#1
  },
  /notebook/calendar/.style 2 opt={draw}{
    elem=[{inner sep=3pt,#1}]{
      \small
      \def\nb@ifcal{
          if (Saturday,Sunday) [/notebook/calendar/weekend]
          if (Monday,day of month=1) { \nb@calendar@kwcode }
          if (at least=\nb@startdate) {}
          else [nodes={text width=2ex,align=right,/notebook/calendar/past}]
          if (between=\nb@startdate and \nb@startdate+6) [nodes={/notebook/calendar/next week}]
        }
      \pgfcalendarcalc[nb@]{#2}{\nb@startdate}
      \calendar[dates=\nb@Y-\nb@M-01 to \nb@Y-\nb@M-last,
                /notebook/calendar/every calendar] (this month) \nb@ifcal;
      \pgfcalendarcalc[nb@]{\nb@Y-\nb@M-last+1}{\nb@tmp}
      \calendar[dates=\nb@Y-\nb@M-01 to \nb@Y-\nb@M-last,
                /notebook/calendar/every calendar] (next month)
                at (6.5cm,0) \nb@ifcal;
    },
  }
}

\csdef{nb@weekday@0}{Mo}
\csdef{nb@weekday@1}{Tu}
\csdef{nb@weekday@2}{We}
\csdef{nb@weekday@3}{Th}
\csdef{nb@weekday@4}{Fr}
\csdef{nb@weekday@5}{\bfseries Sa}
\csdef{nb@weekday@6}{\bfseries Su}

\def\nb@parsemonth#1#2{
  \pgfcalendarcalc[nb@]{#1}{\nb@tmp}
  \pgfcalendar{cal}{\nb@Y-\nb@M-01}{\nb@Y-\nb@M-last}{
    \listxadd{#2}{{\pgfcalendarcurrentday}{\pgfcalendarcurrentweekday}}
  }
}

\pgfkeys{
  /notebook/month list/.code 2 args={
    \scriptsize%
    \nbkeys{
      elem={[draw]\node at (.5\elemwidth,0){\strut \Y{} - \pgfcalendarmonthname{\M}};},
      columns={Date/1,#2}
    }
    \def\nb@cols{}
    \foreach \field/\width in {#2} {
      \xdef\nb@cols{\nb@cols,/\width}
    }
    % Collect all days in month
    \let\alldays\relax
    \nb@parsemonth{#1}{\alldays}
    \def\notebook@month@day##1##2##3{
      \pgfkeys{
        /notebook/.cd,
        columns={%
          {\rlap{\csuse{nb@weekday@##3}}\hphantom{Mit} \pgfcalendarmonthshortname{\M} ##2}/1%
          ##1%
        },
      }
    }
    \def\do##1{\expandafter\notebook@month@day\expandafter{\nb@cols}##1}
    \dolistloop{\alldays}
  },
}


\newcommand{\notebookpage}[2][]{%
  \thispagestyle{empty}%
  \begin{tikzpicture}[remember picture,overlay]\sffamily%
    \pgfkeys{/notebook/.cd,canvas start=current page text area,#2}%
  \end{tikzpicture}%
  \clearpage
}
